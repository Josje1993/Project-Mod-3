\documentclass[12pt]{article}
\usepackage[dutch]{babel}

\usepackage{amsmath}
\usepackage{graphicx} 



\author{Joshua de Bie - s1442627\\Thomas Raaijen - s1462431\\Josje van 't Padje - s1423037\\Gerwin Puttenstein - s1487779}
\date{\today}
\title{Integrated Project Netwerk Systemen}

\begin{document}
\maketitle
\thispagestyle{empty}
\setcounter{page}{0}
\newpage

\tableofcontents
\newpage

\section{Introductie}
In dit artikel wordt het integrated project besproken.\\ Met het integrated project is het de bedoeling om een chatapplicatie te ontwerpen dat het Ad hoc netwerk setup implementeert. Het onderwerp chatapplicaties is interessant, omdat in de afgelopen tijd Whatsapp vaak in het nieuws kwam na de overname door Facebook. Whatsapp zou veel gebruikers hebben verloren aan een andere chatapplicatie genaam "Telegram". Whatsapp verloor zoveel gebruikers, omdat deze gebruikers bang waren om hun privacy te verliezen. Daarom is het van belang dat er een chatapplicatie komt die gebruik maakt van (as)symmetrische sleutels, om de privacy van zijn gebruikers te waarborgen. In dit artikel word de gemaakte chatapplicatie gepresenteerd die extensief gebruik zal maken van de bekende beveiligingsmethoden.\\
Dit artikel wordt opgebouwd met het welbekende IMRAD-structuur, dit wil zeggen dat eerste de ge\"implementeerde ontwerp keuzes en methoden aan bod komen en daarna de de resultaten worden gepresenteerd en daaraan een korte conclusie gegeven wordt. Wij zullen ons eerst kort introduceren aan de hand van onze functie binnen het project. Vervolgens zal het ontwerp van onze chatapplicatie aanbod komen met onze keuzes als het gaat om het interne netwerk systeem, daarna zal er over gegaan worden naar het grafische ontwerp aanbod komen. De resultaten van beiden ontwerpen zullen besproken en zal er een korte conclusie getrokken worden. 
\newpage

\section{Functies binnen het team}
\textbf{Gerwin} \\
\emph{Lead Designer Network Systems} \\
Gerwin zorgt ervoor dat het netwerk gedeelte van het project structuur krijgt en dat de rest van het team hiermee aan de slag kan zonder alles uit te moeten zoeken.
\\

\noindent\textbf{Joshua} \\
\emph{Lead Designer Software Systems} \\
Joshua zorgt ervoor dat het software gedeelte van het project goed gestructureerd wordt. Dit wil zeggen dat de klassen van het project overzichtelijk en voorzien worden van het nodige commentaar.
\\

\noindent\textbf{Josje} \\
\emph{Spokeswoman} \\
Josje is het vliegende teamlid, zij gaat iedereen ondersteunen waar nodig is en houd een oogje in het zeil. 
\\

\noindent\textbf{Thomas} \\
\emph{Chairman} \\
Thomas zorgt voor alle verslagen en dat iedereen zijn of haar taak uitvoert. Zelf zou hij overal bijvallen en de besprekingen leiden. Ook moet hij ervoor zorgen dat er een koers wordt gevaren. Ook zal hij de overzicht in de drive moeten waarborgen.
\newpage

\section{verantwoordelijkheden}
\textbf{Gerwin} \\
Gerwin is verantwoordelijk voor de network systemen. Dit wil zeggen dat hij degene die moet kunnen uitleggen hoe het systeem zou moeten werken en het idee over brengen aan de rest van de groep.
\\

\noindent\textbf{Joshua} \\
Joshua is verantwoordelijk voor de software systemen. Dit wil zeggen dat hij degene die moet kunnen uitleggen hoe het systeem zou moeten werken en het idee over brengen aan de rest van de groep. Ook zorgt hij ervoor dat de klassen goed gedocumenteerd zijn en dat het op te leveren software de nieuwste versie en de beste versie is.
\\

\noindent\textbf{Josje} \\
Josje moet dus een oogje in het zeil wordt gehouden. Als het gaat om verantwoordelijkheden heeft Josje er geen maar zij zorgt ervoor dat de verantwoordelijkheden van anderen ook worden voldaan.
\\

\noindent\textbf{Thomas} \\
Thomas is verantwoordelijk voor de alle documenten dat ingeleverd moeten worden. Ook is hij verantwoordelijk dat de taken binnen de groep worden uitgevoerd en dat iedereen ergens mee aan de slag kan gaan tijdens het gehele project. Net zoals Josje springt hij bij met tijdens het project.
\newpage

\section{Netwerk Ontwerp}


\subsection{ADHOC Implementatie}


\subsection{Ondersteuning Dataverkeer}


\subsection{UDP Headers}


\subsection{Multi Socket Klassen}


\subsection{ALIVE}


\subsection{Link Betrouwbaarheid En Packet Verlies}


\subsection{Beveiliging}

\newpage

\section{Grafische Ontwerp}


\subsection{Schetsen}


\subsection{ChatGUI versie 3(ChGv3)}

\subsection*{Chat Window}

\subsection*{Close Window en Settings Window}


\subsection{ConnectGUI versie 4(CGv4)}
Profiel ondersteuning.
\subsection*{Connect Window}

\subsection*{Create Profile Window}

\newpage

\section{Resultaten}

\subsection{Netwerk Ontwerp}

\subsection{Grafisch Ontwerp}

\newpage

\section{Conclusie}

\newpage

\section{Nawoord}

\bibliographystyle{plain}
\bibliography{ProjectBib}
\end{document}
